\section{Omówienie algorytmu}
	W pracy posłużyliśmy się dwoma najpopularniejszymi algorytmami minimalno-odległościowymi:
	
	\begin{itemize}
		\item NM(ang. \textit{nearest mean}) - najbliższej średniej,
		\item KNN(ang. \textit{k nearest neighbors}) - k najbliższych sąsiadów
	\end{itemize}

	Ideą tych algorytmów jest klasyfikacji danego osobnika do odpowiadającej mu klasy na podstawie minimalnych odległości do pewnych elementów swojego otoczenia.

	\textbf{Najbliższej centroidy} - dla każdej klasy na podstawie wszystkich obiektów, wyliczana jest średnia centroida. Obiekt przypisywane jest na podstawie minimalnej odległości do tej centroidy.
	
	\textbf{K najbliżyszch sąsiadów} - ze znanych obiektów wybieramy k najbliższych klasyfikowanemu obiektowi. Nowy obiekt zostaje przypisany do klasy, w której znajduje się najwięcej z k reprezentantów.
	