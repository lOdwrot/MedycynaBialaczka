\section{Ranking cech}
	Selekcja cech dla danych ma 3 najważniejsze zalety:
	
	\begin{itemize}
		\item \textbf{Redukcja przeładowania} - mniej niepotrzebnych/nieznaczących danych oznacza podejmowanie decyzji w oparciu o znaczące dane, a nie o \textit{szum},
		\item \textbf{Zwiększenie celności} - mniej mylących danych oznacza lepszą celność/dokładność,
		\item \textbf{Redukcja czasu trenowania} - mniej danych = szybszy algorytm.
	\end{itemize}
	
	\bigbreak
	Jaki jednak algorytm rankingowy dla cech zastosować?
	
	\begin{itemize}
		\item \textbf{Testy Statystyczne} - sprawdzenie zależności poszczególnej cechy a wynikowej klasy w oparciu o test statystyczny np. test chi-kwadrat,
		\item \textbf{Rekurencyjna eliminacja cech} - sprawdzenie zależności poprzez budowanie modelu, przeprowadzeniu statystyk, zbudowaniu ponownie modelu z mniejszą ilością cech i porównanie statystyk. Im lepsze statystyki tym lepszy zestaw cech. Metoda czasochłonna.,
		\item \textbf{Analiza głównych składowych} - interpretacja zbioru jako chmury N-punktów(obserwacji) w przestrzeni K-wymiarowej(zmiennych). Zadaniem algorytmy jest takie obrócenie układu współrzędnych by maksymalizować wariancję pierwszej współrzędnej, a następnie kolejnych.
	\end{itemize}
	
	\bigbreak
	Dla naszych danych przeprowadziliśmy ranking cech przy pomocy testu statystycznego chi-kwadrat. W tabeli \ref{3} przedstawiliśmy wynik rankingu cech dla danych znormalizowanych i nie.
	
	\begin{table}[h!]
		\centering
		\begin{tabular}{|| c | c ||} 
			\hline
			Dane znormalizowane & Dane nieznormalizowane \\ [0.5ex] 
			\hline\hline
			Stan pobudzenia szpiku & Miejsce krwawienia \\ 
			Anemia & Stan pobudzenia szpiku \\
			Gałka oczna & Poziom limfocytów \\
			Obniżenie poziomu RBC & Anemia \\
			Główne komórki szpiku & Obniżenie poziomu RBC \\
			Poziom limfocytów & Główne komórki szpiku \\
			Liczba płytek krwi & Gałka oczna \\
			Wrażliwość mostka & Liczba płytek krwi \\
			Powiększenie jąder & Wrażliwość mostka \\
			Reakcja & Powiększenie jąder \\
			Poziom WBC & Reakcja \\
			Stopień krwawienia & Poziom WBC \\
			Powiększenie węzłów chłonnych & Uszkodzenie w sercu \\
			Uszkodzenie w sercu & Ból kości \\
			Ból kości & Powiększenie węzłów chłonnych \\
			Centralny układ nerwowy & Niedojrzałe komórki \\
			Miejsce krwawienia & Centralny układ nerwowy \\
			Niedojrzałe komórki & Powiększenie wątroby\\
			Powiększenie wątroby & Stopień krwawienia \\
			Temperatura & Temperatura \\ [1ex] 
			\hline
		\end{tabular}
		\caption{Ranking cech}
		\label{3}
	\end{table}