\section{Ranking cech}
	\blindtext
	The table \ref{table:1} is an example of referenced \LaTeX elements.
	\begin{table}[h!]
		\centering
		\begin{tabular}{|| c | c | c ||} 
			 \hline
			 Nr cechy & Nazwa cechy & Możliwe wartości \\ [0.5ex] 
			 \hline\hline
			 1 & Temperatura & 1, 2 \\ 
			 2 & Anemia & 1, 2, 3 \\
			 3 & Stopień krwawienia & 1, 2 \\
			 4 & Miejsce krwawienia & 1, 2, 3, 4, 5, 6, 7, 8 \\
			 5 & Bóle kości & 1, 2 \\
			 6 & Wrażliwość mostka & 1, 2 \\
			 7 & Powiększenie węzłów chłonnych & 1, 2 \\
			 8 & Powiększenie wątroby & 1, 2 \\
			 9 & Centralny układ nerwowy & 1, 2 \\
			 10 & Powiększenie jąder & 1, 2 \\
			 11 & Uszkodzone serce & 1, 2 \\
			 12 & Gałka oczna & 1, 2 \\
			 13 & Poziom WBC & 1, 2, 3 \\
			 14 & Poziom RBC & 1, 2, 3 \\
			 15 & Płytk krwi & 1, 2 \\
			 16 & Niedojrzałe komórki & 1, 2 \\
			 17 & Pobudzenie szpiku & 1, 2, 3 \\
			 18 & Główne komórki & 1, 2, 3 \\
			 19 & Poziom limfocytów & 1, 2, 3 \\
			 20 & Reakcja & 1, 2 \\ [1ex] 
			 \hline
		\end{tabular}
		\caption{Opis cech}
		\label{table:1}
	\end{table}

	\begin{table}[h!]
		\centering
		\begin{tabular}{||c | c ||} 
			 \hline
			 Nr & Jednostk chorobowa \\ [0.5ex] 
			 \hline\hline
			 1 & L1 - type \\ 
			 2 & L2 - type \\
			 3 & L3 - type \\
			 4 & Undifferentation \\
			 5 & Differentation in part \\
			 6 & Granuocylosisi \\
			 7 & Granua mononucleaw \\
			 8 & Mononucleacyble \\
			 9 & Redikaukemia \\
			 10 & Subatue grandblacyta \\
			 11 & Granulacytarna \\
			 12 & Lymphocytia \\
			 13 & Granue mononclea \\
			 14 & Mononuclea \\
			 15 & Lymphosarcoma leukemia \\
			 16 & Pamacea Leukemia \\
			 17 & Multicapilary be laukemia \\
			 18 & Acicople granulotyne leukemia \\
			 19 & Basaphi granulocyte leukemia \\
			 20 & Macronuclacycle teuekema \\ [1ex] 
			 \hline
		\end{tabular}
		\caption{Jednostki chorobowe}
		\label{table:1}
	\end{table}