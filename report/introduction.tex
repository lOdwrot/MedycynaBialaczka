\section{Wstęp}

Praca składa się z omówienia zastosowanego algorytmu, definicji problemu, opisania cech i przedstawieniu ich rankingu, samego badania/eksperymentu wraz z opisem wpływu zmiennych na jakość wyników, prezentacji i omówienia rezultatów. 
\\
\\
Wykorzystanie systemów uczących się w diagnostyce medycznej nie jest zjawiskiem nowym. Dane medyczne od dawna są przechowywane w celu ponownego wykorzystania w przypadku wystąpienia podobnych objawów wśród różnych pacjentów. W tym celu dane przechodzą obróbkę, zostają sklasyfikowane i ujednolicone. Tak zbudowany system, nawet w obrębie pojedynczego szpitala jest w stanie wytworzyć ilość danych pozwalającą na zbudowanie i \textit{wytrenowanie} dowolnego klasyfikatora \cite{1}.